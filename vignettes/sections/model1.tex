We restricted ourselves to analysing plot means.
The phenotypic value $Y_{ijk}$ for variable $Y$, germplasm $i$, environment $j$ and block $k$ was modelled as :

\begin{equation}
	Y_{ijk} = \mu_{ij} + \beta_{jk} + \varepsilon_{ijk} ; \quad \varepsilon_{ijk} \sim \mathcal{N} (0,\sigma^2_{j}),
	\label{model1}
\end{equation}

where
$\mu_{ij}$ was the mean of germplasm $i$ in environment $j$ (note that this parameter, which corresponds to an entry, confounds the population effect and the population $\times$ environment effect);
$\beta_{jk}$ was the effect of block $k$ in environment $j$ satisfying the constraint\footnote{Note that it is quite different from \citet{riviere_hierarchical_2015} where the model was done only for two blocks. Here there is no restriction on the number of blocks.} $\sum\limits_{k=1}^K \beta_{jk} = 1$ ;
$\varepsilon_{ijk}$ was the residual error;
$\mathcal{N} (0,\sigma^2_{j})$ denoted normal distribution centred on 0 with variance $\sigma^2_{j}$, which was specific to environment $j$.

We took advantage of the similar structure of the trials on each environment of the network to assume that trial residual variances came from a common distribution :

\begin{displaymath}
	\sigma^2_{j} \sim \frac{1}{Gamma(\nu,\rho)},
\end{displaymath}

where $\nu$ and $\rho$ are unknown parameters.
Because of the low number of residual degrees of freedom for each farm, we used a hierarchical approach in order to assess mean differences on farm.
For that, we placed vague prior distributions on the hyperparameters $\nu$ and $\rho$ :

\begin{displaymath}
	\nu \sim Uniform(\nu_{min},\nu_{max}) ; \quad \rho \sim Gamma(10^{-6},10^{-6}).
\end{displaymath}


In other words, the residual variance of a trial within environment was estimated using all the informations available on the network rather than using the data from that particular trial only.

The parameters $\mu_{ij}$ and $\beta_{j1}$ were assumed to follow vague prior distributions~:

\begin{displaymath}
	\mu_{ij} \sim \mathcal{N}(\mu_{.j},10^{6}); \quad \beta_{j1} \sim \mathcal{N}(0,10^{6}).
\end{displaymath}


The inverse gamma distribution has a support bounded by 0 (consistent with the definition of a variance) and may have various shapes including asymmetric distributions.
From an agronomical point of view, the assumption that trial variances were heterogeneous was consistent with organic farming: there were as many environments as farmers leading to a high heterogeneity.
Environment was here considered in a broad sense: practices (sowing date, sowing density, tilling, etc.), pedo climatic conditions, biotic and abiotic stress, \dots \citep{desclaux_changes_2008}.
Moreover, the inverse gamma distribution had conjugate properties that facilitated MCMC convergence.
This model was therefore a good choice based on both agronomic and statistical criteria.

The residual variance estimated from the controls was assumed to be representative of the residual variance of the other entries.
Blocks were included in the model only if the trial had blocks.



