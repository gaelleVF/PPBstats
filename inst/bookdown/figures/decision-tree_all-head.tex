\documentclass[tikz, border=10pt]{standalone}
% cf http://cloford.com/resources/colours/websafe1.htm
\definecolor{sv-all}{RGB}{255,153,255}
\definecolor{sv-gxe-m2}{RGB}{255,204,255}
\definecolor{sv}{RGB}{255,153,255}
\definecolor{m1}{RGB}{153,153,255}
\definecolor{m2}{RGB}{153,255,153}
\definecolor{ge}{RGB}{255,255,153}
\definecolor{sp}{RGB}{255,153,153}
\definecolor{vi}{RGB}{255,204,153}
\definecolor{ma}{RGB}{204,255,153}
\definecolor{hedo}{RGB}{102,102,204}
\definecolor{nap}{RGB}{102,204,102}
\definecolor{ca}{RGB}{255,51,0}
\definecolor{half}{RGB}{153,153,0}

\definecolor{level-1}{RGB}{255,204,255}
\definecolor{level-2}{RGB}{153,153,255}
\definecolor{level-3}{RGB}{153,255,153}
\definecolor{level-4}{RGB}{255,255,153}
\definecolor{level-5}{RGB}{255,153,153}
\definecolor{level-6}{RGB}{255,204,153}
\definecolor{level-7}{RGB}{204,255,153}

\usetikzlibrary{
  mindmap,
  decorations.pathreplacing,    % paths with shoapes of curly braces
  positioning,     % positions like above of node
  fit              % legend bounding box fitting all nodes
}
\tikzset{
  node distance=4ex and 4ex,
  % on grid,  % node distance from the centers
  every node/.style = {
    rectangle,
    minimum width=5em,
    minimum height=3ex,
    text depth=1pt,
    draw,
    outer sep = 2pt,
    inner sep = 3pt
  },
  every edge/.style = {->,draw},
  virtual/.append style = {draw=none, circle, minimum width=1em},
  % virtual/.append style = {draw, color=black!50},   % debugging purposes
  several-all/.append style = {fill = sv-all},
  several-gxe-m2/.append style = {fill = sv-gxe-m2},
  m1/.append style = {fill = m1},
  m2/.append style = {fill = m2},
  gxe/.append style = {fill = ge},
  sp/.append style = {fill = sp},
  vi/.append style = {fill = vi},
  ma/.append style = {fill = ma},
  aux/.append style = {fill = none},
  hedo/.append style = {fill = hedo},
  nap/.append style = {fill = nap},
  ca/.append style = {fill = ca},
  half/.append style = {fill = half},
  legendkey/.append style = {minimum width=3ex},
  legendtext/.append style = {draw=none, fill = black!10},
  ->,        % arrows for all
  >=stealth  % arrow type
}
\pgfdeclarelayer{background}
\pgfsetlayers{background,main}


\begin{document}

\begin{tikzpicture}[grow cyclic, align=flush center,
every node/.style=concept, concept color=teal!40, font=\huge, text width=5cm,
every node/.append style={scale=1},
%every concept/.style={rectangle},
    level 1/.style={level distance=11cm, sibling angle=72, concept color=level-1, font=\huge, text width=7cm},
    level 2/.style={level distance=11cm, sibling angle=70, concept color=level-2, font=\huge, text width=4cm},
    level 3/.style={level distance=10cm, sibling angle=50, concept color=level-3, font=\huge, text width=4cm},
    level 4/.style={level distance=10cm, sibling angle=40, concept color=level-4, font=\huge, text width=4cm},
    level 5/.style={level distance=12cm, sibling angle=40, concept color=level-5, font=\huge, text width=6cm},
    level 6/.style={level distance=12cm, sibling angle=40, concept color=level-6, font=\huge, text width=7cm},
    level 7/.style={level distance=12cm, sibling angle=40, concept color=level-7, font=\huge, text width=5cm},
    ]
\node{\textbf{Experimental designs and statistical methods for PPB}}
child{ node {\textbf{Compare different varieties evaluated for selection in different locations}}
  child{ node {Analysis of agronomic traits}
  }
  child{ node {Sensory analysis}
  }
}
child { node {\textbf{Improve the prediction of a target variable for selection}}
  child { node {Analysis of agronomic traits}
  }
}
child { node {\textbf{Study diversity structure and identify parents to cross based on either good complementarity or similarity for some traits}}
  child { node {Analysis of agronomic traits}
    }
  child { node {Analysis of molecular data}
  }
}
child { node {\textbf{Study network of seed circulation}}
  child{ node {Analysis of network topology}
  }
}
child{ node {\textbf{Study the response of varieties under selection over several environments}}
  child{ node {Analysis of agronomic traits}
  }
}
\end{tikzpicture}
\end{document}


